%%%%%%%%%%%%%%%%%%%%%%%%%%%%%%%%%%%%%%%%%
% Medium Length Professional CV
% LaTeX Template
% Version 2.0 (8/5/13)
%
% This template has been downloaded from:
% http://www.LaTeXTemplates.com
%
% Original author:
% Trey Hunner (http://www.treyhunner.com/)
%
% Important note:
% This template requires the resume.cls file to be in the same directory as the
% .tex file. The resume.cls file provides the resume style used for structuring the
% document.
%
%%%%%%%%%%%%%%%%%%%%%%%%%%%%%%%%%%%%%%%%%

%----------------------------------------------------------------------------------------
%	PACKAGES AND OTHER DOCUMENT CONFIGURATIONS
%----------------------------------------------------------------------------------------

\documentclass{resume} % Use the custom resume.cls style

\usepackage[left=0.75in,top=0.6in,right=0.75in,bottom=0.6in]{geometry} % Document margins
\usepackage{mathptmx}
\usepackage{hyperref}
\name{Horia Alexandru Maior} % Your name
\address{School of Computer Science, University of Nottingham, NG8 1BB, UK} % Your address
\address{horia.maior@nottingham.ac.uk} % Your phone number and email
\address{Mob: +44 7468420350 } % Your phone number and email
\address{\url{http://www.nottingham.ac.uk/~psxhama}}
\begin{document}

%----------------------------------------------------------------------------------------
%	EDUCATION SECTION
%----------------------------------------------------------------------------------------

\begin{rSection}{Education}
{\bf University of Nottingham, UK} \hfill {\em 2012-2016} \\
PhD Candidate at the Mixed Reality Lab \& Computer Science \\
Cross-Discipline Research with Horizon Digital Economy - Centre for Doctoral Training \\
Thesis:\emph{``The physiological measure and feedback of workload using fNIRS''} \smallskip \\
\vspace{-1 mm}

{\bf Swansea University, UK} \hfill {\em June 2012} \\
B.Sc. in Computer Science \\
Dissertation: Modelling and Simulation of Railways –\emph{``Safe Trains - a CSP Approach''}\smallskip \\
Award: First class honours degree

\end{rSection}

%----------------------------------------------------------------------------------------
%	Research interests  SECTION
%----------------------------------------------------------------------------------------
\begin{rSection}{Research Interests}
\begin{rSubsection}{Brain and Physiological data use for Human Computer Interaction}{ }{ }{ }
\vspace{-5 mm}
My research lies in the intersection between Human-Computer Interaction (HCI), Brain-Computer
Interface (BCI), and Human Factors. We are currently looking into the integration of non-invasive
brain monitoring devices, such as functional Near Infrared Spectroscopy (fNIRS), in the field of HCI.
More specifically, I am looking at how we can use such devices to sense physiological responses to
human cognition and mental workload, and how we can use this quantitative measure in real time for HCI lab based evaluation.

\begin{itemize}
  \item Continuous involvement with the HCI community (often at ACM CHI - the top conference for Human-Computer Interaction).
  \item Involved in the Academic Review process for multiple conferences - ACM-CHI 2016, UIST 2016, ACM-CHI 2015, ACM-CHIPLAY 2014, IEEE CASE 2014, IIiX2014.
  \item  Memberships - Association for Computing Machinery (ACM), Chartered Institute of Ergonomics and Human Factors.
\end{itemize}

\end{rSubsection}

\end{rSection}

%----------------------------------------------------------------------------------------
%	Publication SECTION
%----------------------------------------------------------------------------------------
\begin{rSection}{Publications}
\begin{rSubsection}{ }{ }{ }{Peer reviewed Conference Papers, Journal Papers}
\item Lukanov K., \textbf{Maior, H. A.}, and Wilson, M. L. \emph{Using fNIRS in Usability Testing: Understanding the Effect of Web Form Layout on Mental Workload.} In: CHI'16 ACM SIGCHI Conference on Human Factors in Computer Systems, San Jose, California, May 2016. IN PRESS (Acceptance rate 20\%)

\item \textbf{Maior, H. A.} Pike, M., Wilson, M. L., and Sharples, S. \emph{Examining the Reliability of Using fNIRS in Realistic HCI Settings for Spatial and Verbal Tasks.} In: CHI'15 ACM SIGCHI Conference on Human Factors in Computer Systems, Seoul, Korea, April 2015. (Acceptance rate 20\%)

\item \textbf{Maior, H. A.} and Rao, S. (2014.) \emph{A Self-Governing, Decentralized, Extensible Internet of Things To Share Electrical Power Efficiently.} In IEEE CASE 2014 International Conference on Automation Science and Engineering, 18th August 2014. Taipei Taiwan.

\item Pike, M., \textbf{Maior, H. A.}, Porcheron, M., Sharples, S. and Wilson, M. L. (2014). \emph{Measuring the effect of Think Aloud Protocols on Workload using fNIRS.} In: CHI'14 ACM SIGCHI Conference on Human Factors in Computer Systems, April-May 2014, Toronto. (Acceptance rate 20\%)

\item \textbf{Maior, H. A.}, Pike, M., Wilson, M. L., and Sharples, S. \emph{Continuous detection of workload overload: An fNIRS approach.} In Contemporary Ergonomics and Human Factors 2014: Proceedings of the international conference on Ergonomics \& Human Factors 2014, Southampton, UK, April 2014.
\end{rSubsection}

\begin{rSubsection}{ }{ }{ }{Workshop papers, Abstract Papers, Magazine Articles}
    \item \textbf{Maior, H.A.}, and Stringer, P. (2015) \emph{The Values of Games for Health and Well Being}. Interdisciplinary Reflections on Games and Human Values - Workshop , The ACM SIGCHI Annual Symposium on Computer-Human Interaction in Play (CHI PLAY), London, 2015.
    \item \textbf{Maior, H.A.}, Wilson, M.L. and Sharples, S. (2015) \emph{fNIRS in Human Factors.} 2FNIRS Workshop, Toulouse, France April 2015.
    \item \textbf{Maior, H.A.}, Pike, M. \emph{Measuring Work Overload.} The Ergonomist Magazine, May 2014.
    \item \textbf{Maior, H.A.}, Pike, M., Wilson, M.L. and Sharples, S. (2013) \emph{Directly Evaluating the Cognitive Impact of Search User Interfaces: a Two-Pronged Approach with fNIRs.} EuroHCIR 2013 Workshop, Dublin, Ireland, August 2013
\end{rSubsection}

\end{rSection}
\vspace{-2 mm}

%----------------------------------------------------------------------------------------
%	Teaching and Work Experience SECTION
%----------------------------------------------------------------------------------------
\begin{rSection}{Teaching}
    \begin{rSubsection}{Supervision}{2015-2016}{3rd Year Project Supervision}{University of Nottingham}
        \item  ``Evaluating Mental Workload through Psychophysiological Measurements using a Sensory Device''.
    \end{rSubsection}

    \begin{rSubsection}{Supervision}{2014-2015}{Masters Student - HCI Prize for the BEST MASTERS DISSERTATION}{University of Nottingham}
        \item  ``Using fNIRS in Usability Testing: Understanding the Effect of Web Form Layout on Mental Workload''.
    \end{rSubsection}

    \begin{rSubsection}{Lecturing}{2015}{Lectures}{University of Nottingham}
        \item G54MXR Mixed Reality, Lecture: Brain Computer Interfaces for Human Computer Interaction
        \item G51FSE Introduction to Software Engineering
    \end{rSubsection}

    \begin{rSubsection}{Teaching}{Sep 2013 - present}{Teaching Assistant}{University of Nottingham}
        \item G51FSE Introduction to Software Engineering (Java, Object Oriented Programming) (2015-2016)
        \item G54PRG Programming, University of Nottingham (C/C++)
        \item G51CSA Computer Systems Architecture (ARM Assembly Language)
        \item G51FSE Introduction to Software Engineering (2014-2015)
    \end{rSubsection}
\end{rSection}


%----------------------------------------------------------------------------------------
%	Awards and Honours SECTION
%----------------------------------------------------------------------------------------
\begin{rSection}{Awards, Honours and Grants}
    \begin{rSubsection}{``Helf Elf''}{September 2015}{Postgraduate Ingenuity Prize for BEST New Business Idea}{}
        \item \pounds 5000 prize from CityCare Nottingham.
        \item \pounds 1000 Grant from Santander.
        \item Co-Founded Healthy Research CIC.
    \end{rSubsection}
\vspace{-4 mm}
    \begin{rSubsection}{\vspace{-2 mm}}{2015-2019}{Unilever Grant}{www.unilever.co.uk}
        \item Research Grant for one PhD studentship on ``Using fNIRS in Human-Computer Interaction Research''.
        \item With Dr Max L. Wilson, University of Nottingham.
    \end{rSubsection}
%\vspace{-1 mm}
\newpage
    \begin{rSubsection}{}{}{Research Travel Grants}{}
        \item \pounds 1000 Travel Grant from the School of Computer Science, University of Nottingham. (2016)
        \item \pounds 1000 Travel Grant from the School of Computer Science, University of Nottingham. (2015)
        \item \pounds 800 University of Nottingham - Postgraduate Travel Grant Competition. (2014)
    \end{rSubsection}
\vspace{-4 mm}
    \begin{rSubsection}{}{2013 - 2015}{Summer Schools}{}
        \item Digital Economy Web Science and Big Data Analytics, July 2015.
        \item Digital Economy Summer School: Digital Revolutions, Oxford, July 2013.
    \end{rSubsection}
\vspace{-4 mm}
    \begin{rSubsection}{}{2012 - present}{Programming Coach, Contestant, Organizer}{}
        \item  ACM-ICPC Coach at NWERC 2015 - Linköping, Sweden - {http://www.nwerc.eu/}
        \item  ACM-ICPC Site Director at UKIEPC2015 - Nottingham, UK - {http://ukiepc.info/}
        \item  ACM-ICPC Coach at NWERC 2013 - Delft, Netherlands  - {http://2013.nwerc.eu/}
        \item Contestant at the 2012 ACM ICPC - NWERC in Bremen, Germany - {http://2012.nwerc.eu/}
    \end{rSubsection}

\end{rSection}

\vspace{-2 mm}

%----------------------------------------------------------------------------------------
%	TECHNICAL STRENGTHS SECTION
%----------------------------------------------------------------------------------------

\begin{rSection}{Technical Strengths}

\begin{tabular}{ @{} >{\bfseries}l @{\hspace{6ex}} l }
Computer Languages & Java, JavaScript, AngularJS, PHP, C\#, C/C++, MySQL etc\\
Programming environments & Visual Studio, NetBeans, Eclipse, Webstorm, PHPStorm\\
Professional Software & Latex Document Writing, Statistics SPSS, Adobe Illustrator \\
Hardware & EEG, fNIRS Brain Scanner, Arduino, Raspberry Pi
\end{tabular}
\end{rSection}

\vspace{-1 mm}

%----------------------------------------------------------------------------------------
%	Internship  and Work Experience SECTION
%----------------------------------------------------------------------------------------

\begin{rSection}{Internships and Work Experience}
\vspace{-1 mm}
    \begin{rSubsection}{RatedDoctor.com}{Feb 2015 - present}{Chief Technology Officer}{Nottingham}
    \vspace{-1 mm}
        \item \url{www.rateddoctor.com}
        \item Online platform to bring patients and specialists doctors together.
    \end{rSubsection}
\vspace{-1 mm}
\begin{rSubsection}{IIIT-Bangalore and Horizon Digital Economy Research Hub}{Jan 2014 - May 2014}{Research Intern}{Electronic City, Bangalore, India}
\vspace{-1 mm}
\item Digital Economy Research based in Bangalore, India.
\item Project: The INTERNET of THINGS and the Smart Grid.
\item Publication: \textbf{Maior, H. A}. and Rao, S.  \emph{``A Self-Governing, Decentralized, Extensible Internet of Things To Share Electrical Power Efficiently. In IEEE CASE 2014''}
\end{rSubsection}
\vspace{-1 mm}
\begin{rSubsection}{Invensys Rail and Swansea University}{June 2011 - Sep 2011}{Research Intern}{Swansea, UK}
\vspace{-1 mm}
\item Modelling Railways Safety using formal language CSP (communicating sequential processes)
\item Time simulating capacity of Railways using formal language Timed CSP.
\end{rSubsection}

%---------------------------------k
\vspace{-1 mm}

  \begin{rSubsection}{AMPsuite LTD}{Sep 2014 - June 2015}{Web developer}{Nottingham}
        \item Building web-based software tools for Records Label companies across the globe.
        \item Client-focused, understand client needs.
        \item Front end, back end - PHP, JavaScript, MySQL, html5.
    \end{rSubsection}
    \vspace{-1 mm}
    %------------------------------------------------
    \begin{rSubsection}{Amazon.co.uk}{June 2010 - June 2012}{Warehouse Problem Solver}{Swansea}
        \item  Find and correct mistakes in the system.
        \item  Roam the warehouse with a laptop on wheels and offer feedback on how to do things better.
\end{rSubsection}

\end{rSection}
%----------------------------------------------------------------------------------------


\end{document}
