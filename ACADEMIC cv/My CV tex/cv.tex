%%%%%%%%%%%%%%%%%%%%%%%%%%%%%%%%%%%%%%%%%
% Medium Length Professional CV
% LaTeX Template
% Version 2.0 (8/5/13)
%
% This template has been downloaded from:
% http://www.LaTeXTemplates.com
%
% Original author:
% Trey Hunner (http://www.treyhunner.com/)
%
% Important note:
% This template requires the resume.cls file to be in the same directory as the
% .tex file. The resume.cls file provides the resume style used for structuring the
% document.
%
%%%%%%%%%%%%%%%%%%%%%%%%%%%%%%%%%%%%%%%%%

%----------------------------------------------------------------------------------------
%	PACKAGES AND OTHER DOCUMENT CONFIGURATIONS
%----------------------------------------------------------------------------------------

\documentclass{resume} % Use the custom resume.cls style

\usepackage[left=0.75in,top=0.6in,right=0.75in,bottom=0.6in]{geometry} % Document margins
\usepackage{mathptmx}
\usepackage{hyperref}
\name{Horia Alexandru Maior} % Your name
\address{School of Computer Science, University of Nottingham, NG8 1BB, UK} % Your address
%\address{horia.maior@nottingham.ac.uk} % Your phone number and email
\address{horia-alexandru.maior@nottingham.edu.cn} % Your phone number and email
%\address{Mob: +44 7468420350 } % Your phone number and email
\address{\url{http://www.nottingham.ac.uk/~psxhama}}
\begin{document}

%----------------------------------------------------------------------------------------
%	EDUCATION SECTION
%----------------------------------------------------------------------------------------

\begin{rSection}{Education}
{\bf University of Nottingham, UK} \hfill {\em 2012-2017} \\
Multidisciplinary PhD Researcher at the Mixed Reality Lab \& Computer Science \\
Horizon Digital Economy Research Centre for Doctoral Training \\
Thesis:\emph{``Real-Time Physiological Measure and Feedback of Workload using fNIRS''} \smallskip \\
\vspace{-1 mm}

{\bf Swansea University, UK} \hfill {\em June 2012} \\
B.Sc. in Computer Science \\
Dissertation: Modelling and Simulation of Railways Signalling –\emph{``Safe Trains - a CSP Approach''}\smallskip \\
Award: First class honours degree

\end{rSection}

%----------------------------------------------------------------------------------------
%	Research interests  SECTION
%----------------------------------------------------------------------------------------
\begin{rSection}{Research Interests}
\begin{rSubsection}{Brain and Physiological data use for Human Computer Interaction}{ }{ }{ }
\vspace{-5 mm}
My research lies in the intersection between Human-Computer Interaction (HCI), Brain-Computer Interface (BCI), and Human Factors. I am looking into the integration of non-invasive brain and physiological sensors, such as functional Near Infrared Spectroscopy (fNIRS), in the field of HCI. More specifically, I am interested in how we can use such devices to sense physiological responses to human cognition and mental workload, and use this continuous, quantitative measure, as an additional channel of information about the user during interaction.

\begin{itemize}
  \item Papers presented at CHI2016, NeuroErgonomics2016, CHI2015, CHIPLAY2015, CHI2014, IEEE RAS 2014, EHF2014.
  \item Involved in the Academic Review process for multiple conferences - Associate Chair at ACM-CHI 2017, ACM-CHI 2016, UIST 2016, ACM-CHI 2015, ACM-CHIPLAY 2014, IEEE CASE 2014, IIiX2014.
  \item  Memberships - Association for Computing Machinery (ACM), Chartered Institute of Ergonomics and Human Factors.
\end{itemize}

\end{rSubsection}

\end{rSection}

%----------------------------------------------------------------------------------------
%	Publication SECTION
%----------------------------------------------------------------------------------------
\begin{rSection}{Publications}
\begin{rSubsection}{ }{ }{ }{Peer reviewed Conference Papers, Journal Papers}
	
\item \textbf{Maior, H.A.}, Wilson, M.L., Sharples, S. \emph{Workload Alerts - Using Physiological Measures of Mental Workload to Provide Feedback during Tasks}. (Under Review) ToCHI2017 Transactions on Computer-Human Interaction.

\item Muralidharan, A., \textbf{Maior, H.A.}, and Rao, S. \emph{A self-governing and decentralized network of smart objects to share electrical power autonomously}. (2016) IN PRESS: Springer Open Journal on Infrastructure Complexity.
 	
\item Lukanov K., \textbf{Maior, H. A.}, and Wilson, M. L. \emph{Using fNIRS in Usability Testing: Understanding the Effect of Web Form Layout on Mental Workload.} In: CHI'16 ACM SIGCHI Conference on Human Factors in Computer Systems, San Jose, California, May 2016.(Acceptance rate 20\%)

\item \textbf{Maior, H. A.} Pike, M., Wilson, M. L., and Sharples, S. \emph{Examining the Reliability of Using fNIRS in Realistic HCI Settings for Spatial and Verbal Tasks.} In: CHI'15 ACM SIGCHI Conference on Human Factors in Computer Systems, Seoul, Korea, April 2015. (Acceptance rate 20\%)

\item \textbf{Maior, H. A.} and Rao, S. (2014.) \emph{A Self-Governing, Decentralized, Extensible Internet of Things To Share Electrical Power Efficiently.} In IEEE CASE 2014 International Conference on Automation Science and Engineering, 18th August 2014. Taipei Taiwan.

\item Pike, M., \textbf{Maior, H. A.}, Porcheron, M., Sharples, S. and Wilson, M. L. (2014). \emph{Measuring the effect of Think Aloud Protocols on Workload using fNIRS.} In: CHI'14 ACM SIGCHI Conference on Human Factors in Computer Systems, April-May 2014, Toronto. (Acceptance rate 20\%)

\item \textbf{Maior, H. A.}, Pike, M., Wilson, M. L., and Sharples, S. \emph{Continuous detection of workload overload: An fNIRS approach.} In Contemporary Ergonomics and Human Factors 2014: Proceedings of the international conference on Ergonomics \& Human Factors 2014, Southampton, UK, April 2014.
\end{rSubsection}

\begin{rSubsection}{ }{ }{ }{Workshop papers, Abstract Papers, Magazine Articles}
	\item \textbf{Maior, H.A.}, Sharples, S., and Wilson, M.L. \emph{Subjective and Objective Methods to Continuously Monitor Workload}. Neuroergonomics 2016: The brain at work and in everyday life, Paris, France 2016.
    \item \textbf{Maior, H.A.}, and Stringer, P. (2015) \emph{The Values of Games for Health and Well Being}. Interdisciplinary Reflections on Games and Human Values - Workshop , The ACM SIGCHI Annual Symposium on Computer-Human Interaction in Play (CHI PLAY), London, 2015.
    \item \textbf{Maior, H.A.}, Wilson, M.L. and Sharples, S. (2015) \emph{fNIRS in Human Factors.} 2FNIRS Workshop, Toulouse, France April 2015.
    \item \textbf{Maior, H.A.}, Pike, M. \emph{Measuring Work Overload.} The Ergonomist Magazine, May 2014.
    \item \textbf{Maior, H.A.}, Pike, M., Wilson, M.L. and Sharples, S. (2013) \emph{Directly Evaluating the Cognitive Impact of Search User Interfaces: a Two-Pronged Approach with fNIRs.} EuroHCIR 2013 Workshop, Dublin, Ireland, August 2013
\end{rSubsection}

\end{rSection}
\vspace{-2 mm}

%----------------------------------------------------------------------------------------
%	Teaching and Work Experience SECTION
%----------------------------------------------------------------------------------------

\begin{rSection}{Teaching and Academic Related Experience}
	\begin{rSubsection}{Teaching Fellow}{Feb 2017 - present}{Computer Science}{University of Nottingham Ningbo Campus UNNC}
			\item Teach Computer Science Subjects at all Levels;
			\item Develop Course and  Exams Materials;
			\item Develop Computer Science Modules Curriculums;
			\item Supervise Undergraduate and Postgraduate Students (UK campus);
			\item Perform Administrative duties as required;
	\end{rSubsection}
	
	\begin{rSubsection}{Postdoctoral Research Fellow}{Sep 2016 - Feb 2017}{Computer Science}{Horizon Digital Economy Research Hub, University of Nottingham, UK}
		\item Wayward Project -  This project investigated the use of Computer Science techniques that capture doctors and healthcare professionals' workload in hospitals, in order to learn and improve the out of hours secondary care, tasking and tasking allocation for doctors and medical professionals, and generally the efficiency of out of hours healthcare in the UK.
	\end{rSubsection}	
	
	\begin{rSubsection}{Supervision}{2014-2016}{Student Project Supervision}{University of Nottingham}
        \item  ``Evaluating Mental Workload through Psycho-physiological Measurements using a Sensory Device''. Final year Undergraduate project
        \item  ``Using fNIRS in Usability Testing: Understanding the Effect of Web Form Layout on Mental Workload''. Masters Thesis - HCI Prize for the BEST MASTERS DISSERTATION
    \end{rSubsection}
    \vspace{-1 mm}
\vspace{-1 mm}
    \begin{rSubsection}{Guest Lecturing}{Jan 2015 - Feb 2017}{Lectures}{University of Nottingham}
    	\item Introduction to Computer Programming 
        \item G54MXR Mixed Reality, Lecture: Brain Computer Interfaces for Human Computer Interaction
        \item G51FSE Introduction to Software Engineering
    \end{rSubsection}
\vspace{-1 mm}
    \begin{rSubsection}{Teaching}{Sep 2013 - Feb 2017}{Teaching Assistant}{University of Nottingham}
        \item G51FSE Introduction to Software Engineering (Java, Object Oriented Programming) (2015-2016)
        \item G54PRG Programming, University of Nottingham (C/C++)
        \item G51CSA Computer Systems Architecture (ARM Assembly Language)
        \item G51FSE Introduction to Software Engineering (2014-2015)
    \end{rSubsection}
\vspace{-1 mm}
    \begin{rSubsection}{Outreach experience}{March 2016}{International Officer for the School of Computer Science}{University of Nottingham}
        \item Create campaigns to meet and attract excellent potential undergraduate and postgraduate students to join our Computer Science programmes at Nottingham University.
        \item Attending universities fairs abroad (\url{http://en.riuf.ro/}), to present to schools and potential students.
    \end{rSubsection}
\vspace{-1mm}
    \begin{rSubsection}{Student Engagement}{2012 - 2017}{Programming Coach, Programming Contest Organizer}{}
    	\item Every year I get involved in organizing a few programming contests, both, locally (University of Nottingham), and national (UK).
    	\item  ACM-ICPC Site Director at UKIEPC2016 and UKIEPC2015 - Nottingham, UK - {http://ukiepc.info/}
    	\item  ACM-ICPC Coach at NWERC 2015 (Linköping, Sweden), NWERC 2013 (Delft, Netherlands) - {http://2013.nwerc.eu/} 
    	\item Contestant at the 2012 ACM ICPC - NWERC in Bremen, Germany - {http://2012.nwerc.eu/}
    \end{rSubsection}
\end{rSection}
\vspace{-2 mm}

%----------------------------------------------------------------------------------------
%	Awards and Honours SECTION
%----------------------------------------------------------------------------------------
\begin{rSection}{Awards, Honours and Grants}
    \begin{rSubsection}{``Helf Elf''}{September 2015}{Postgraduate Ingenuity Prize for BEST New Business Idea}{}
        \item \pounds 5000 prize from CityCare Nottingham.
        \item \pounds 1000 Grant from Santander.
        \item Co-Founded Healthy Research CIC.
    \end{rSubsection}
%\vspace{-4 mm}
%    \begin{rSubsection}{}{2015-2019}{Unilever Grant}{www.unilever.co.uk}
%        \item Research Grant for one PhD studentship on ``Using fNIRS in Human-Computer Interaction Research''.
%        \item With Dr Max L. Wilson, University of Nottingham.
%    \end{rSubsection}
%\vspace{-1 mm}
%\newpage
    \begin{rSubsection}{}{}{Research Travel Grants}{}
        \item \pounds 1000 Travel Grant from the School of Computer Science, University of Nottingham. (2016)
        \item \pounds 1000 Travel Grant from the School of Computer Science, University of Nottingham. (2015)
        \item \pounds 800 University of Nottingham - Postgraduate Travel Grant Competition. (2014)
    \end{rSubsection}
%\vspace{-4 mm}
    \begin{rSubsection}{}{2013 - 2015}{Summer Schools}{}
        \item Digital Economy Web Science and Big Data Analytics, July 2015.
        \item Digital Economy Summer School: Digital Revolutions, Oxford, July 2013.
    \end{rSubsection}
%\vspace{-4 mm}

\end{rSection}

%\vspace{-2 mm}

%----------------------------------------------------------------------------------------
%	Personal Attributes
%----------------------------------------------------------------------------------------

\begin{rSection}{Personal Attributes}
	\begin{itemize}
		\item Excellent Communication Skills;
	\vspace{-1mm}
		\item Ability to work with others as a team;
	\vspace{-1mm}
		\item Ability to work collaboratively in a multidisciplinary environment;
	\vspace{-1mm}
		\item Ability to work effectively in a multi-cultural environment;
	\vspace{-1mm}
		\item Ability to work to deadlines and to prioritise tasks;
	\vspace{-1mm}
		\item Ability to play a key, team member role in the department I work for;
	\end{itemize}

\end{rSection}

%\vspace{-1 mm}


%----------------------------------------------------------------------------------------
%	TECHNICAL STRENGTHS SECTION
%----------------------------------------------------------------------------------------

\begin{rSection}{Technical Strengths}

	\begin{tabular}{ @{} >{\bfseries}l @{\hspace{6ex}} l }
	Computer Languages & Java, JavaScript, AngularJS, PHP, C\#, C/C++, MySQL etc\\
	Programming environments & Visual Studio, NetBeans, Eclipse, Webstorm, PHPStorm\\
	Professional Software & Latex Document Writing, Statistics SPSS, Adobe Illustrator \\
	Hardware & EEG, fNIRS Brain Scanner, Arduino, Raspberry Pi
	\end{tabular}
\end{rSection}

%\vspace{-1 mm}

%----------------------------------------------------------------------------------------
%	Internship  and Work Experience SECTION
%----------------------------------------------------------------------------------------

\begin{rSection}{Industry Work Experience, Internships, and Volunteering}
%\vspace{-1 mm}
    \begin{rSubsection}{RatedDoctor.com}{Feb 2015 - present}{Chief Technology Officer}{Nottingham}
%    \vspace{-1 mm}
        \item \url{www.rateddoctor.com}
        \item Online platform to bring patients and specialists doctors together.
        \item Responsibilities included determining the scope for the project and developing requirements - working closely with the CEO -, project management - working closely with the designers and development team.
    \end{rSubsection}
%\vspace{-1 mm}

	\begin{rSubsection}{ACMCHI Conference on  Computer-Human Interaction}{May 2016}{Student Volunteer at chi4good}{San Jose, California}
%    \vspace{-1 mm}
        \item Getting involved in preparing and helping out with successfully running the top conference in Computer-Human Interaction. 
        \item Duties included coordinating the plenary and other conference talks, preparing materials for Workshops, conference registration, and of course meeting many people and having lots of fun!
    \end{rSubsection}

	\begin{rSubsection}{IIIT-Bangalore and Horizon Digital Economy Research Hub}{Jan 2014 - May 2014}{Research Intern}{Electronic City, Bangalore, India}
	%\vspace{-1 mm}
		\item Digital Economy Research based in Bangalore, India.
		\item Project: The INTERNET of THINGS and the Smart Grid.
		\item Publication: \textbf{Maior, H. A}. and Rao, S.  \emph{``A Self-Governing, Decentralized, Extensible Internet of Things To Share Electrical Power Efficiently. In IEEE CASE 2014''}
	\end{rSubsection}
	%\vspace{-1 mm}
	\begin{rSubsection}{Invensys Rail and Swansea University}{June 2011 - Sep 2011}{Research Intern}{Swansea, UK}
		%\vspace{-1 mm}
		\item Modelling Railways Safety using formal language CSP (communicating sequential processes)
		\item Time simulating capacity of Railways using formal language Timed CSP.
	\end{rSubsection}

%---------------------------------k
%\vspace{-1 mm}

  \begin{rSubsection}{AMPsuite LTD}{Sep 2014 - June 2015}{Web developer}{Nottingham}
        \item Building web-based software tools for Records Label companies across the globe.
        \item Client-focused, understand client needs.
        \item Front end, back end - PHP, JavaScript, MySQL, html5.
    \end{rSubsection}
%    \vspace{-1 mm}
    %------------------------------------------------
    \begin{rSubsection}{Amazon.co.uk}{June 2010 - June 2012}{Warehouse Problem Solver}{Swansea}
        \item  Coordinating a small team, ensuring the quality of Amazon services 
        \item  Dealing with customer enquiries and solve problems
	\end{rSubsection}

\end{rSection}
%----------------------------------------------------------------------------------------


\end{document}
