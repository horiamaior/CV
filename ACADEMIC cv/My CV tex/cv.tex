%%%%%%%%%%%%%%%%%%%%%%%%%%%%%%%%%%%%%%%%%
% Medium Length Professional CV
% LaTeX Template
% Version 2.0 (8/5/13)
%
% This template has been downloaded from:
% http://www.LaTeXTemplates.com
%
% Original author:
% Trey Hunner (http://www.treyhunner.com/)
%
% Important note:
% This template requires the resume.cls file to be in the same directory as the
% .tex file. The resume.cls file provides the resume style used for structuring the
% document.
%
%%%%%%%%%%%%%%%%%%%%%%%%%%%%%%%%%%%%%%%%%

%----------------------------------------------------------------------------------------
%	PACKAGES AND OTHER DOCUMENT CONFIGURATIONS
%----------------------------------------------------------------------------------------

\documentclass{resume} % Use the custom resume.cls style

\usepackage[left=0.75in,top=0.6in,right=0.75in,bottom=0.6in]{geometry} % Document margins
\usepackage{mathptmx}
\usepackage{hyperref}
\name{Horia Alexandru Maior} % Your name
\address{School of Computer Science, University of Nottingham, NG8 1BB, UK} % Your address
\address{horia.maior@nottingham.ac.uk} % Your phone number and email
%\address{horia-alexandru.maior@nottingham.edu.cn} % Your phone number and email
\address{Mob: +44 7468420350 } % Your phone number and email
\address{\url{https://www.nottingham.ac.uk/~psahm2/}}
\begin{document}

%----------------------------------------------------------------------------------------
%	EDUCATION SECTION
%----------------------------------------------------------------------------------------

\begin{rSection}{Education}
{\bf University of Nottingham, UK} \hfill {\em 2012-2017} \\
Multidisciplinary PhD in Digital Economy with the Mixed Reality Lab \& School of Computer Science \\
Horizon Digital Economy Research Centre for Doctoral Training \\
Thesis:\emph{``Real-Time Physiological Measure and Feedback of Workload''}\\
Disciplines: Human Computer Interaction (HCI), Brain Computer Interfaces (BCI), Computer Science.
\vspace{-1 mm}

{\bf Swansea University, UK} \hfill {\em June 2012} \\
B.Sc. in Computer Science \\
Dissertation: Modelling and Simulation of Railways Signalling –\emph{``Safe Trains - a CSP Approach''}\smallskip \\
Award: First class honours degree

\end{rSection}

%----------------------------------------------------------------------------------------
%	Research interests  SECTION
%----------------------------------------------------------------------------------------
\begin{rSection}{Research Interests}
\begin{rSubsection}{Brain and Physiological data use for Human Computer Interaction}{ }{ }{ }
\vspace{-5 mm}

I am a multidisciplinary researcher based in the Mixed Reality Lab at the University of Nottingham, UK. As a result of the Horizon Centre for Doctoral Training PhD, my research is uniquely placed, sitting between Computer Science, Human-Computer Interaction (HCI) and Brain Computer Interfaces. 

In 2017 I have secured the prestigious EPSRC Doctoral Prize Award which has offered me the support for my 2-years proposed research agenda based in the Mixed Reality Laboratory at the University of Nottingham. The Prize is an opportunity for the most outstanding EPSRC-funded PhD graduates to receive up to two years of additional support following their PhD and kick start their own independent research career. Subjects covered include Science, Engineering, Technology and Medicine within the EPSRC remit.

 The winning project, \textit{Brain-Tracker: A fit-bit for the Brain based on physiological sensors, tracking the mental workload of everyday tasks} explores the scientific grounding for an inevitable future technology: a Mental Fitbit (MFB) for the Brain.

%My research lies in the intersection between Human-Computer Interaction (HCI), Brain-Computer Interface (BCI), and Human Factors. I am looking into the integration of non-invasive brain and physiological sensors, such as functional Near Infrared Spectroscopy (fNIRS), in the field of HCI. More specifically, I am interested in how we can use such devices to sense physiological responses to human cognition and mental workload, and use this continuous, quantitative measure, as an additional channel of information about the user during interaction.

\begin{itemize}
	\item Recently \textbf{visited and presented} at the MIT Media Lab, Air Lab in Drexel University, Conquir Lab in Drexel University, School of Computer Science at the San Francisco University, the HCI Lab at Tufts University.
	\item I have  \textbf{presented} at venues such as CHI 2018, fNIRS2018, CHI 2016, CHI 2015, CHIPLAY 2015, CHI 2014, NeuroErgonomics 2016, 2FNIRS 2015, IEEE RAS 2014, EHF 2014.
	\vspace{-2mm}
	\item Often involved in the review and senior review process for multiple conferences and journals - I have been awarded the \textbf{Excellent Reviewer Award} - I have reviewed for CHI 2019, CHI 2018, UIST 2018, 2017, 2016 International Journal of Human-Computer Studies, CHI-PLAY 2017, ACM-CHI 2016 (\textbf{Senior Reviewer} for LBW), UIST 2016, ACM-CHI 2015, ACM-CHIPLAY 2014, IEEE CASE 2014, and IIiX 2014.
	\vspace{-2mm}
	\item Always involved into \textbf{organising Academic Conferences} - I was the \textbf{Associate Chair} for Courses at the ACM-CHI 2018 where I have previously acted as a Student Volunteer in 2017 and 2016.
	\vspace{-2mm}
	\item  \textbf{Memberships} - Association for Computing Machinery (ACM), Chartered Institute of Ergonomics and Human Factors, fNIRS Society.
\end{itemize}

\end{rSubsection}

\end{rSection}
	\vspace{-3mm}
%----------------------------------------------------------------------------------------
%	Publication SECTION
%----------------------------------------------------------------------------------------
\begin{rSection}{Publications}
\begin{rSubsection}{ }{ }{ }{Peer reviewed Conference Papers, Journal Papers}

\item Benerradi, J. \textbf{Maior, H.A.}, Marinescu, A., Clos, J., Sharples, S., Wilson, M.L. \emph{Adventures in Real-Time Machine Learning of Mental Workload using fNIRS Data}.  Submitted to CHI2019 (Under Review).
\item Midha, S., Wilson, M.L., \textbf{Maior, H.A.}, Sharples, S. \emph{Detecting Variation in Mental Workload within Everyday Work Tasks with fNRIS}  Submitted to CHI2019 (Under Review).
\item \textbf{Maior, H.A.}, Wilson, M.L., Sharples, S. \emph{Workload Alerts - Using Physiological Measures of Mental Workload to Provide Feedback during Tasks}.  ToCHI2018 Transactions on Computer-Human Interaction.
\item Lukanov K., \textbf{Maior, H. A.}, and Wilson, M. L. \emph{Using fNIRS in Usability Testing: Understanding the Effect of Web Form Layout on Mental Workload.} In: CHI'16 ACM SIGCHI Conference on Human Factors in Computer Systems, San Jose, California, May 2016.(Acceptance rate 20\%)
\item \textbf{Maior, H. A.} Pike, M., Wilson, M. L., and Sharples, S. \emph{Examining the Reliability of Using fNIRS in Realistic HCI Settings for Spatial and Verbal Tasks.} In: CHI'15 ACM SIGCHI Conference on Human Factors in Computer Systems, Seoul, Korea, April 2015. (Acceptance rate 20\%)
\item \textbf{Maior, H. A.} and Rao, S. (2014.) \emph{A Self-Governing, Decentralized, Extensible Internet of Things To Share Electrical Power Efficiently.} In IEEE CASE 2014 International Conference on Automation Science and Engineering, 18th August 2014. Taipei Taiwan.
\item Pike, M., \textbf{Maior, H. A.}, Porcheron, M., Sharples, S. and Wilson, M. L. (2014). \emph{Measuring the effect of Think Aloud Protocols on Workload using fNIRS.} In: CHI'14 ACM SIGCHI Conference on Human Factors in Computer Systems, April-May 2014, Toronto. (Acceptance rate 20\%)
\item \textbf{Maior, H. A.}, Pike, M., Wilson, M. L., and Sharples, S. \emph{Continuous detection of workload overload: An fNIRS approach.} In Contemporary Ergonomics and Human Factors 2014: Proceedings of the international conference on Ergonomics \& Human Factors 2014, Southampton, UK, April 2014.
\end{rSubsection}

\begin{rSubsection}{ }{ }{ }{Workshop papers, Abstract Papers, Magazine Articles}
	\item \textbf{Maior, H.A.}, Ramchurn, R., Cai, M., Wilson, M.L., Martindale, S., Benford, S.  \emph{fNIRS and Neurocinematics.}  The fNIRS 2018 conference, Tokyo, Japan, October 2018.
	\item Midha, S., Wilson, M.L., \textbf{Maior, H.A.}, Sharples, S. \emph{Detecting Variation in Mental Workload Levels within Everyday Work Tasks using fNIRS.}  The fNIRS 2018 conference, Tokyo, Japan, October 2018.
	\item \textbf{Maior, H.A.}, Wilson, M.L., Locke, C., Swann, D.  \emph{Brain activity and mental workload associated with artistic practice}. CHI 2018 Artistic Brain Computer Interfaces Workshop. SIGCHI Conference on Human Factors in Computer Systems, Montreal 2018.
	\item Wilson, M.L., Sharon, N., \textbf{Maior, H.A.}, Midha, S., Craven, M.P.  \emph{Mental workload as personal data: designing a cognitive activity tracker}. CHI 2018 Mental Health Simphosium. SIGCHI Conference on Human Factors in Computer Systems, Montreal 2018..
	\item Wilson, M.L., Alsuraykh N., \textbf{Maior, H.A.},  \emph{Measuring mental workload in IIR user studies with fNIRS}. Conference on Information Retrieval 2017.
	\item \textbf{Maior, H.A.}, Sharples, S., and Wilson, M.L. \emph{Subjective and Objective Methods to Continuously Monitor Workload}. Neuroergonomics 2016: The brain at work and in everyday life, Paris, France 2016.
    \item \textbf{Maior, H.A.}, and Stringer, P. (2015) \emph{The Values of Games for Health and Well Being}. Interdisciplinary Reflections on Games and Human Values - Workshop , The ACM SIGCHI Annual Symposium on Computer-Human Interaction in Play (CHI PLAY), London, 2015.
    \item \textbf{Maior, H.A.}, Wilson, M.L. and Sharples, S. (2015) \emph{fNIRS in Human Factors.} 2FNIRS Workshop, Toulouse, France April 2015.
    \item \textbf{Maior, H.A.}, Pike, M. \emph{Measuring Work Overload.} The Ergonomist Magazine, May 2014.
    \item \textbf{Maior, H.A.}, Pike, M., Wilson, M.L. and Sharples, S. (2013) \emph{Directly Evaluating the Cognitive Impact of Search User Interfaces: a Two-Pronged Approach with fNIRs.} EuroHCIR 2013 Workshop, Dublin, Ireland, August 2013
\end{rSubsection}

\end{rSection}
\vspace{-2 mm}

%----------------------------------------------------------------------------------------
%	Teaching and Work Experience SECTION
%----------------------------------------------------------------------------------------

\begin{rSection}{Academic Related Experience}
	
	\begin{rSubsection}{EPSRC Doctoral Prize Award}{Nov 2017 - Nov 2019}{The Mixed Reality Lab}{University of Nottingham }
		\item The EPSRC Doctoral Prize scheme supports fellowships of two years' duration for exceptional researchers who have recently finished an EPSRC-funded PhD.
		\item Winning Project: \textit{Brain-Tracker: A fit-bit for the Brain based on physiological sensors, tracking the mental workload of everyday tasks.}
		\item As part of this fellowship, am an associate researcher in the Mixed Reality Laboratory conducting multidisciplinary research that contributes the scientific grounding for an inevitable future technology: a Mental Fitbit for the brain. 
		\item My research is placed between Human-Computer Interaction, Brain Computer Interfaces, and Human Factors and I am focused on using brain and physiological data collected using non-invasive wearable devices, in order to understand and track the amount of stress and mental workload we go through during day-to-day activities.
		\item Part of this fellowship, I spent some time as a visiting scholar for 1 month, visiting top laboratories in the field around the US, including: the MIT Media Lab, Tufts University, the Air Lab and Conquir Lab in Drexel University and the Department of Computer Science at the University of San Francisco. I used this time to present research, expand my network, and build new collaborations with top researchers in the field.
	\end{rSubsection}
	
	\begin{rSubsection}{Teaching Fellow in Computer Science}{Feb 2017 - Sep 2017}{Computer Science}{University of Nottingham Ningbo, CHINA}
			\item Module coordinator for two modules: AE1FSE Software Engineering and AE3COM Computability.
			\item I supervised undergraduate and postgraduate student projects (both UK and China campus);
			\item Responsible of administrative duties; I was the Social Chair of the department, responsible for organizing Student-Staff activities outside the teaching hours. I organized a few movie nights and planned for a monthly `Show and Tell' - where the invited academic presents in 30 minutes their passion for CS to the students. 
	\end{rSubsection}
	
%	\begin{rSubsection}{Visiting Researcher}{Feb 2017 - Sep 2017}{Mixed Reality Lab}{University of Nottingham UK}
%		\item while I was in China, I still kept connected with the Mixed Reality Lab and the UK Campus.
%		\item I have supervised two Practice Led Projects with two PhD students.
%		\item I have worked remotely on the Brain Computer Interface projects with researchers in the Mixed Reality Lab.
%		\item Been involved in writing academic papers with the build up network of researchers in the Computer Science Department in Nottingham.		
%	\end{rSubsection}
	
	\begin{rSubsection}{Research Assistent}{Sep 2016 - Feb 2017}{Computer Science}{Horizon Digital Economy Research Hub, University of Nottingham, UK}
		\item I was part of \textbf{the Wayward Project} -  This project was focused on understanding doctors and healthcare professionals' work in hospitals contexts - and in particular their use of different technology (COW's, mobile phone usage etc.), with the ultimate goal of reducing their workload by facilitating technology that better fits their needs.
		\item part of this role, I was observing doctors during out-of-hours care in the two major hospitals in Nottingham, and monitored and analysed doctors interaction with different technologies.
	\end{rSubsection}	
	
	\begin{rSubsection}{Students Supervision}{2014-2018}{Student Project Supervision}{University of Nottingham}
		\item I have supervised Practive Led Projects of PhD students, some of which transformed into research publications.
		\item I have supervised Masters students, in particular one of them was awarded the HCI Prize for the best dissertation in the School.  His thesis was then transformed into a strong research publication ``Using fNIRS in Usability Testing: Understanding the Effect of Web Form Layout on Mental Workload''.
		\item But I have also supervised undergraduate dissertations and group projects.
    \end{rSubsection}
    \vspace{-1 mm}
    
	\begin{rSubsection}{Guest Lecturer}{Accross Multiple Institutions}{}{}
    	\item I am often invited to give a guest lectures at Nottingham University in various modules, including Software Engineering, Introduction to Programming, Mixed Reality, and Brain Cpmputer Interfaces.
    	\item I have also delivered a guest lecture on ``Brain Computer Interfaces for Human Computer Interaction'' at Drexel University, University of San Francisco and University of Nottingham Ningbo Campus.
    \end{rSubsection}
\vspace{-1 mm}
    \begin{rSubsection}{Teaching}{Sep 2013 - Feb 2017}{Teaching Assistant}{University of Nottingham}
        \item Introduction to Software Engineering (Java, Object Oriented Programming) 
        \item Programming, University of Nottingham (C/C++)
        \item Computer Systems Architecture (ARM Assembly Language)
  %      \item Introduction to Computer Programming with c\#        

    \end{rSubsection}
\vspace{-1 mm}
    \begin{rSubsection}{Outreach experience}{March 2016}{International Officer for the School of Computer Science}{University of Nottingham}
        \item Attended campaigns to meet and attract potential undergraduate and postgraduate students to join our Computer Science programmes at Nottingham University.
        \item Attending universities fairs abroad (\url{http://en.riuf.ro/}), presenting to schools and having outreach with prospective students.
    \end{rSubsection}
\vspace{-1mm}
    \begin{rSubsection}{Student Engagement}{2012 - 2017}{Programming Coach, Programming Contest Organizer}{}
    	\item Every year I get involved in organizing a few programming contests for the students at the University. Typically we run two, a local contest focused on the students from the University of Nottingham, and helping out with the second, a month later, which is the national UK based contest.
    	\item  ACM-ICPC Site Director at UKIEPC2018, UKIEPC2017, UKIEPC2016 and UKIEPC2015 - Nottingham, UK - {http://ukiepc.info/}
    	\item  We then take the best teams from our institution, we coach them and take them to the European regional ACM-ICPC: NWERC 2017 (Bath, UK), NWERC 2016 (Bath, UK), NWERC 2015 (Linköping, Sweden), NWERC 2013 (Delft, Netherlands) - {http://2013.nwerc.eu/} 
    	\item I used to be a contestant during my undergraduate degree, at the 2012 ACM ICPC - NWERC in Bremen, Germany - {http://2012.nwerc.eu/}
    \end{rSubsection}
\end{rSection}
\vspace{-4 mm}

%----------------------------------------------------------------------------------------
%	Awards and Honours SECTION
%----------------------------------------------------------------------------------------
\begin{rSection}{Awards, Honours and Grants}
	  
	  \begin{rSubsection}{EPSRC Doctoral Prize Award}{Nov 2017 - Nov 2019}{The Mixed Reality Lab}{University of Nottingham }
		\item \pounds £70 000 doctoral prize by the UK Engineering and Physical Sciences Research Council.
		\item This was offered towards self-funding my current research possition.
		\item Winning proposal was \textit{Brain-Tracker: A fit-bit for the Brain based on physiological sensors, tracking the mental workload of everyday tasks.}
	\end{rSubsection}


    \begin{rSubsection}{``Helf Elf''}{September 2015}{Postgraduate Ingenuity Prize for BEST New Business Idea}{University of Nottingham}
        \item \pounds 5000 Prize from CityCare Nottingham.
        \item \pounds 1000 Grant from Santander.
        \item Co-Founded Healthy Research CIC.
    \end{rSubsection}
%\vspace{-4 mm}
%    \begin{rSubsection}{}{2015-2019}{Unilever Grant}{www.unilever.co.uk}
%        \item Research Grant for one PhD studentship on ``Using fNIRS in Human-Computer Interaction Research''.
%        \item With Dr Max L. Wilson, University of Nottingham.
%    \end{rSubsection}
%\vspace{-1 mm}
%\newpage
\vspace{-4mm}
    \begin{rSubsection}{}{}{Research Travel Grants}{ University of Nottingham}
        \item \pounds 1000 Travel Grant offered by the School of Computer Science in 2016.
        \item \pounds 1000 Travel Grant offered by the School of Computer Science in 2015.
        \item \pounds 800 Postgraduate Travel Grant Competition offered by the Postgraduate School in 2014
    \end{rSubsection}
\vspace{-4 mm}
    \begin{rSubsection}{}{2013 - 2015}{Summer Schools}{}
        \item Digital Economy Web Science and Big Data Analytics, July 2015.
        \item Digital Economy Summer School: Digital Revolutions, Oxford, July 2013.
    \end{rSubsection}
%\vspace{-4 mm}

\end{rSection}

\vspace{-4 mm}

%----------------------------------------------------------------------------------------
%	Personal Attributes
%----------------------------------------------------------------------------------------

\begin{rSection}{Personal Attributes}
	\begin{itemize}
		\item Excellent Communication Skills;
	\vspace{-3mm}
		\item Ability to work with others as a team;
	\vspace{-3mm}
		\item Ability to work collaboratively in a multidisciplinary environment;
	\vspace{-3mm}
		\item Ability to work effectively in a multi-cultural environment;
	\vspace{-3mm}
		\item Ability to work to deadlines and to prioritise tasks;
	\vspace{-3mm}
		\item Ability to play a key, team member role in the department I work for;
	\end{itemize}

\end{rSection}

%\vspace{-1 mm}


%----------------------------------------------------------------------------------------
%	TECHNICAL STRENGTHS SECTION
%----------------------------------------------------------------------------------------

\begin{rSection}{Technical Strengths}

	\begin{tabular}{ @{} >{\bfseries}l @{\hspace{6ex}} l }
	Computer Languages & Java, PHP, C\#, C/C++ etc\\
	Programming environments & Visual Studio, Eclipse, Webstorm, PHPStorm\\
	Professional Software & Latex Document Writing, Overleaf, Statistics SPSS, Adobe Illustrator \\
	Equipment & Brain and physiological sensing using EEG, fNIRS, Thermal Cameras, Heart and respirathory based measurements, Empatica, others.
	\end{tabular}
\end{rSection}

%\vspace{-1 mm}

%----------------------------------------------------------------------------------------
%	Internship  and Work Experience SECTION
%----------------------------------------------------------------------------------------

\begin{rSection}{Industry Work, Internships, and Volunteering}
%\vspace{-1 mm}
    \begin{rSubsection}{RatedDoctor.com}{2015 - present}{Chief Technology Officer}{Nottingham, UK}
%    \vspace{-1 mm}
        \item \url{www.rateddoctor.com}
        \item Rateddoctor.com is an location based Online platform that brings patients and specialists doctors together.
		\item Sitting between the CEO and the development team in order to control the technological development at RatedDoctor.
		\item Overseeing innovation in rated doctor, heavily involved in R\&D.
		\item Transforming knowledge and research into impact on people's lives.
    \end{rSubsection}
%\vspace{-1 mm}

	\begin{rSubsection}{ACMCHI Conference on Computer-Human Interaction}{May 2017 - May 2018}{Courses Co-Chair}{Montreal, CANADA}
	%    \vspace{-1 mm}
	\item Organizing Courses at the CHI conference.
	\item Duties included coordinating the format, the program, the submission and generally all issues related to Courses at CHI.
	\item And of course meeting many people and having lots of fun!
\end{rSubsection}

\begin{rSubsection}{ACMCHI Conference on Computer-Human Interaction}{May 2016}{Student Volunteer at chi4good}{San Jose, CALIFORNIA}
	%    \vspace{-1 mm}
	\item Getting involved in preparing and helping out with successfully running the top conference in Computer-Human Interaction. 
	\item Duties included coordinating the plenary and other conference talks, preparing materials for Workshops, conference registration, and of course meeting many people and having lots of fun!
\end{rSubsection}

%---------------------------------k
%\vspace{-1 mm}

  \begin{rSubsection}{AMPsuite LTD}{Sep 2014 - June 2015}{Part-time Web developer}{Nottingham}
  		\item \url{www.ampsuite.com}
        \item Building web-based software tools for Records Label companies across the globe.
        \item Client-focused, understand client needs.
        \item Front end, back end - PHP, JavaScript, MySQL, html5.
    \end{rSubsection}
%    \vspace{-1 mm}

	\begin{rSubsection}{IIIT-Bangalore and Horizon Digital Economy Research Hub}{Jan 2014 - May 2014}{Research Intern}{Electronic City, Bangalore, INDIA}
	%\vspace{-1 mm}
	\item Digital Economy Research based in Bangalore, India.
	\item Project: The INTERNET of THINGS and the Smart Grid.
	\item Publication: \textbf{Maior, H. A}. and Rao, S.  \emph{``A Self-Governing, Decentralized, Extensible Internet of Things To Share Electrical Power Efficiently. In IEEE CASE 2014''}
\end{rSubsection}
%\vspace{-1 mm}
	
\begin{rSubsection}{Invensys Rail and Swansea University}{June 2011 - Sep 2011}{Research Intern - undergraduate}{Swansea, UK}
	%\vspace{-1 mm}
	\item Modelling Railways Safety using formal language CSP (communicating sequential processes)
	\item Time simulating capacity of Railways using formal language Timed CSP.
\end{rSubsection}
    %------------------------------------------------
 %   \begin{rSubsection}{Amazon.co.uk}{June 2010 - June 2012}{Warehouse Problem Solver}{Swansea}
       % \item  Coordinating a small team, ensuring the quality of Amazon services 
        %\item  Dealing with customer enquiries and solve problems
%	\end{rSubsection}

\end{rSection}
%----------------------------------------------------------------------------------------


\end{document}
